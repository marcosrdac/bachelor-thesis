%%%%%%%%%%%%%%%%%%%%%%%%%%%%%%%%%%%%%%%%%%%%%%%%%%%%%%%%%%%%%%%%%%%
%                                                                 %
%                           BIBLIOGRAFIA                          %
%                                                                 %
%%%%%%%%%%%%%%%%%%%%%%%%%%%%%%%%%%%%%%%%%%%%%%%%%%%%%%%%%%%%%%%%%%%
% O exemplo abaixo é o recomendado pois utiliza o bibtex que
% assegura uma lista de referências completa e sem falta.
%
% Entretanto, para aqueles que continuam usando a montagem manual,
% este arquivo deve iniciar-se com
%	\chapter*{Referências Bibliográficas}
%	\addcontentsline{toc}{chapter}{Referências Bibliográficas}
%	\markboth{Referências Bibliográficas}{Referências Bibliográficas}
% 
 
\selectlanguage{brazil}
\bibliographystyle{pppgbib}
%\bibliographystyle{plainnat}

% Note que o argumento do comando \bibliography é longo e, por isso,
% foi dividido em diversas linhas. A presença do % é fundamental 
% para eliminar o "fim de linha"
%\bibliography{bib-article,bib-book,bib-proceeding,bib-collection,bib-thesis,bib-master,bib-tcc,bib-report,bib-manual,bib-misc}
\bibliography{mybib}
