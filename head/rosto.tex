%%%%%%%%%%%%%%%%%%%%%%%%%%%%%%%%%%%%%%%%%%%%%%%%%%%%%%%%%%%%%%%%%%
%                                                                 %
%  Este arquivo exemplo prepara a capa e a folha de rosto da      %
%  monografia. Observe claramente este exemplo, principal-         %
%  mente no que tange as letras maiúsculas e minúsculas.          %
%                                                                 %
%%%%%%%%%%%%%%%%%%%%%%%%%%%%%%%%%%%%%%%%%%%%%%%%%%%%%%%%%%%%%%%%%%%
%%%%%%%%%%%%%%%%%%%%%%%%%%%%%%%%%%%%%%%%%%%%%%%%%%%%%%%%%%%%%%%%%%%%%%%%%
%                                                                       %
% Titulo do trabalho, em maiúsculo e minúsculo                          %
%                                                                       %
%\newcommand{\THESISTITLE}{INVERSÃO DE FORMA COMPLETA DA ONDA (FWI) ACÚSTICA COMO O TREINAMENTO DE UMA REDE NEURAL RECORRENTE (RNN)}
%\newcommand{\thesistitle}{Inversão de forma completa da onda (FWI) acústica como o treinamento de uma rede neural recorrente (RNN)}				              %
\newcommand{\THESISTITLE}{REDES NEURAIS RECORRENTES DE ELMAN APLICADAS À INVERSÃO ACÚSTICA DA FORMA DE ONDA COMPLETA UTILIZANDO DIFERENCIAÇÃO AUTOMÁTICA}
\newcommand{\thesistitle}{Redes neurais recorrentes de Elman aplicadas à inversão acústica da forma de onda completa utilizando diferenciação automática}				              %
%                                                                       %
%%%%%%%%%%%%%%%%%%%%%%%%%%%%%%%%%%%%%%%%%%%%%%%%%%%%%%%%%%%%%%%%%%%%%%%%%
%                                                                       %
% Nome do aluno em duas formas: 1) com apenas as primeiras letras em    %
% maiusculo e 2) totalmente em maiusculo (este ultimo sera' usado       %
% na capa)                                                              %
%
\AUTHOR{MARCOS CONCEIÇÃO}
\author{Marcos Conceição}
\orientador{Orientador: Prof. Dr. Reynam Pestana %\\
%Co-orientador: Prof. Dr. Coorientador de Santana
}
%                                                                       %
%                                                                       %
%%%%%%%%%%%%%%%%%%%%%%%%%%%%%%%%%%%%%%%%%%%%%%%%%%%%%%%%%%%%%%%%%%%%%%%%%
%                                                                       %
% Evidentemente, os comandos \comissao que se seguem definem            %
% os componentes da comissão julgadora. Em geral no mestrado,           %
% três desses comando seriam utilizados.                                %
%                                                                       %
%\comissao{xxxxxxxxxxxxxxxxxxx}                                         %
%                                                                       %
% ASSINATURAS
% Digitalize (scanner) uma cópia tipo xerox de boa qualidade, para
% transformar tudo para preto.  Escolha 300 ou 600 dpi e preto&branco.
% Use um editor grafico (ex.: CorelPhotoPaint) para deixar a figura apenas
% com as assinaturas e as linhas horizonais.  Se as linhas nao estiverem
% horizontais, corrija com a ferramenta girar.
% Salve no formato PNG (ex. assinatura.png).
% As tres linhas abaixo permitem inserir o arquivo png.
% As coordenadas entre parenteses do comando \put(x,y) devem ser modificadas
% a fim de posicionar as assinaturas no local adequado.
% Comente as linhas se nao quiser inserir as assinaturas ou para gerar
% a folha de rosto para ser assinada.
%  modificar aqui ao final!
%   %\renewcommand{\assinatura}{\unitlength=1cm\begin{picture}(0,0)
%   %\put(-11.08,-2.03){\includegraphics{assinatura}}
%   %\end{picture}}
%   
\comissao{\assinatura{}Dr. Reynam C. Pestana}                  %
\comissao{Dr. Edvaldo Suzarthe de Araújo} 
\comissao{MC. Victor Koehne Ramalho} 
%%%%%%%%%%%%%%%%%%%%%%%%%%%%%%%%%%%%%%%%%%%%%%%%%%%%%%%%%%%%%%%%%%%%%%%%%
%                                                                       %
% Os comandos seguintes se referem `as datas. O mes de aprovacao        % 
%  (comando \mesaprovacao) deve estar em maiusculo e aparecera' na      %
%  capa. Observe ainda os dois hifens -- seguidos. Eles devem ser       %
%  mantidos, inclusive os espacos em branco.                            %
%
\mesaprovacao{DEZEMBRO -- 2021}          
\dataaprovacao{06/12/2021}
%
%%%%%%%%%%%%%%%%%%%%%%%%%%%%%%%%%%%%%%%%%%%%%%%%%%%%%%%%%%%%%%%%%%%%%%%%%
% __                         
% \ \   ___ __ _ _ __   __ _ 
%  \ \ / __/ _` | '_ \ / _` |
%   \ \ (_| (_| | |_) | (_| |
%    \_\___\__,_| .__/ \__,_|
%               |_|          
%   __                                           
%  / _| __ _ ____   __ _    ___ __ _ _ __   __ _ 
% | |_ / _` |_  /  / _` |  / __/ _` | '_ \ / _` |
% |  _| (_| |/ /  | (_| | | (_| (_| | |_) | (_| |
% |_|  \__,_/___|  \__,_|  \___\__,_| .__/ \__,_|
%                                   |_|          
%
\capa 
%
%%%%%%%%%%%%%%%%%%%%%%%%%%%%%%%%%%%%%%%%%%%%%%%%%%%%%%%%%%%%%%%%%%%%%%%%%
% __   _   _ _   _                             
% \ \ | |_(_) |_| | ___ _ __   __ _  __ _  ___ 
%  \ \| __| | __| |/ _ \ '_ \ / _` |/ _` |/ _ \
%   \ \ |_| | |_| |  __/ |_) | (_| | (_| |  __/
%    \_\__|_|\__|_|\___| .__/ \__,_|\__, |\___|
%                      |_|          |___/      
%   __                      __       _ _                 _      
%  / _| __ _ ____   __ _   / _| ___ | | |__   __ _    __| | ___ 
% | |_ / _` |_  /  / _` | | |_ / _ \| | '_ \ / _` |  / _` |/ _ \
% |  _| (_| |/ /  | (_| | |  _| (_) | | | | | (_| | | (_| |  __/
% |_|  \__,_/___|  \__,_| |_|  \___/|_|_| |_|\__,_|  \__,_|\___|
%                                                               
%                _        
%  _ __ ___  ___| |_ ___  
% | '__/ _ \/ __| __/ _ \ 
% | | | (_) \__ \ || (_) |
% |_|  \___/|___/\__\___/ 
%                         
\titlepage
%
%%%%%%%%%%%%%%%%%%%%%%%%%%%%%%%%%%%%%%%%%%%%%%%%%%%%%%%%%%%%%%%%%%%%%%%%%
