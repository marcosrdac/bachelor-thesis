%%%%%%%%%%%%%%%%%%%%%%%%%%%%%%%%%%%%%%%%%%%%%%%%%%%%%%%%%%%%%%%%%%%
%                                                                 %
%                         AGRADECIMENTOS                          %
%                                                                 %
%%%%%%%%%%%%%%%%%%%%%%%%%%%%%%%%%%%%%%%%%%%%%%%%%%%%%%%%%%%%%%%%%%%
 
\capitulo{Agradecimentos}
 
  Este trabalho é resultado de um esforço muito superior ao que eu poderia compor sozinho. Neste espaço tentarei elencar aqueles que mais diretamente estão ligados ao meu percurso universitário. Sei que falharei em lembrar de todos e peço perdão de antemão por isto.

  Em primeiro lugar, eu agradeço aos meus pais, por se esforçarem para construir uma família estruturada e por nos serem excelentes exemplos. Agradeço à minha mãe, Antonia Conceição, por me incentivar na caminhada acadêmica e por vibrar com cada conquista minha. Agradeço ao meu pai, Everaldo da Conceição, por seu jeito de educar agindo, sobretudo a ser mais paciente com as fases das pessoas e com minhas próprias falhas. Não tenho dúvida que vocês fizerem todo sacrifício possível por mim e pelos meus irmãos.

  Agradeço a Jamile Conceição, minha irmã, por ser meu espelho desde pequeno e a Bruno Conceição, meu irmão caçula, por se empolgar junto comigo e ser boa companhia. Estive triste em cada momento que concluí o dever de abdicar da presença de vocês e de meus pais ao longo deste trajeto, bem como em cada um dos momentos em que não pude contribuir de forma adequada com a nossa instituição.

  Agradeço imensamente ao meu orientador do trabalho de graduação, Reynam Pestana. Meu ritmo de aprendizado nunca foi tão intenso quanto como nestes últimos anos em que tentei te acompanhar. Obrigado pela paciência, direcionamentos e sermões. Mais de uma vez, seus ``vocês têm que estudar isto aqui...'' mudaram minha forma de pensar sobre o mundo. Agradeço também a Átila Saraiva, meu grande amigo e segundo orientador neste trabalho, quem me introduziu à inversão, à computação de alto desempenho e à migração reversa no tempo, bem como me apresentou Reynam. Reynam dizia ``vocês'', pois eu e meus dois orientadores nos reuníamos semanalmente numa espécie de grupo de estudos, o que foi muito inspirador e tecnicamente amadurecedor para mim. Átila foi muito paciente e preocupado com minha evolução e responsabilização por mim mesmo, desde o IFBA, além de um dos melhores professores que eu já tive!

  Victor Koehne me auxiliou bastante a entender a aplicação das diferenças finitas, e sou muito grato por ele se mostrar tão disposto. Ver sua apresentação num evento do capítulo de Geofísica alguns anos atrás foi bem motivador para mim.

  Agradeço de maneira especial ao professor Luis Felipe de Mendonça, um orientador incrível e um dos melhores seres humanos que eu já conheci. Felipe nunca subestima seus alunos, e trabalhar com ele é energizante como um bom chimarrão! Muito obrigado pela liberdade, pela empolgação pela vida e pelo apoio! Aproveito para agradecer ao meu co-orientador de pesquisa, o professor Carlos Lentini, que me abriu os olhos sobre o funcionamento da academia e sempre se mostrou muito disposto a ajudar aos meus colegas e mim. Agradeço também aos meus companheiros do Laboratório de Oceanografia por Satélites (LOS) por estes dois anos de conquistas incríveis.

  Wilson Figueiró é o professor mais didático que eu já tive e sou muito grato por ter sido seu aluno dentro desta casa. Figueiró mostra que é possível ser extremamente embasado e aplicado ao mesmo tempo, e é sem sombra de dúvidas uma referência para o Marcos de hoje. Além disso, a escrita dos meus primeiros artigos foram sob sua orientação. Agradeço também ao professor Hédison Sato, tanto pela qualidade das aulas, quanto pela preocupação e puxões de orelha que sempre calharam de vir em momentos necessários.

  Eu não sei se viria a trabalhar com sísmica se eu não me apaixonasse pela matéria de ondas durante o curso de Física 2, que foi ministrada pelo professor Thiago Albuquerque. Obrigado por não subestimar seus alunos, Thiago! Agradeço também ao professor Tiago Paes, o qual foi meu primeiro orientador numa bolsa de extensão, e me deu liberdade para investir um tempo estudando campos de pressão, além de sempre ter sido bem solícito. Muito do que aprendi nesse período foi utilizado nos meus estudos de sísmica mais para frente.

  Tenho muito a agradecer ao professor Alexsandro Cerqueira. Alex é muito responsável, um professor exímio e um exemplo de ser humano. Aprendi muito do que já tinha desistido sozinho sob a tutela dele. Fico feliz de considerá-lo hoje um grande amigo também! Todo o resto da turma de 2017 --- Matheus Radamés e Lorena da Silva --- são seus orientandos jutamente a Jonh Brian, que também foi meu colega. Estas são pessoas às quais eu admiro em demasia, e cuja companhia foi sempre muito agradável.

  Agradeço enormemente a Gabriela Brito, minha companheira, pela paciência e apoio nos momentos difíceis e pela vibração cada vitória. Não consigo imaginar minha caminhada até aqui sem você por perto. Agradeço também aos seus pais, Daniela Brito e Elviton dos Santos, pela compreensão e pelas conversas!

  Gostaria de agradecer enormemente a Lucidalva de Assis, minha tia e primeira professora. Seus desafios foram essenciais para que eu me afeiçoasse à Matemática. Tia Luci sempre me incentivou de todas as maneiras possíveis a evoluir. Espero poder retribuir tanto amor! Aproveito para agradecer aos meus tios Ronaldo, Ricardo, Cristina e Maria do Carmo de Assis por me darem exemplo no estudo.

  Gostaria de agradecer à minha prima, Manuela Barreto, bem como a Bruno Bulhões, pelas conversas importantes, pelo pelo exemplo nas ações e pela leveza. Apesar da pequena frequência, conversar com Manu sempre virou chaves na minha cabeça, desde minha primeira infância.

  Agradeço a todo o Capítulo Estudantil de Geofísica da UFBA. Com certeza não teria entrado na modelagem, não fosse a presença do Capítulo na minha vida. Agradeço particularmente pelas conversas com Mariana Sampaio e Juliana Diniz, as quais são ídolos para mim e me estabilizaram ao longo do curso. Agradeço enormemente a Letícia Pires por sua responsabilidade e humanidade, bem como por me ajudar tanto no período em que me sobrecarreguei. Devo muito a você, Leti!

  Sou muito grato ao meu amigo Felipe Duarte. Felipe me aguentou nos meu momentos mais caóticos e sempre me trouxe sentido e calmaria. Não tenho como retribuir o que ele fez por mim. Uma pequena versão sua existe para sempre entre os personagens da minha consciência.

  Aproveito para agradecer pela presença de Felipe Heleno Borges pela amizade desde a adolescência e por Zanoni Neto, meu amigo de infância. Obrigado por estarem sempre prontos para ajudar minhas versões passadas. Eu amo vocês. Agradeço às incríveis outras pessoas de Ardur também.

  Mário Oliveira Luciem foi quem plantou em mim a vontade de ser pesquisador e quem me convenceu das possibilidades. Quem me incentivou a gostar de Física e a me tranquilizou desde a adolescência. Mário foi um dos meus maiores educadores. Muito obrigado, Mário!

  Gostaria de agradecer a Victor Said e Walter G. Neves por serem tão universais, por sempre me ofertarem paz e estarem abertos para ouvir e criticar minhas ideias. Meu percurso teria sido muito mais tortuoso sem os seus conselhos.

  Agradeço a Amanda Oliveira e a Ingrid Queiroz pela amizade e pela paciência nos meus períodos mais difíceis. Vocês me ensinaram a dar o devido valor à vida.

  Sou bastante grato aos meus amigos Milena Leite e Igor Moraes, pessoas que me estabilizaram em diversos momentos e que sempre me incentivaram a ser melhor.

  Por fim, agradeço à minha avó Gildete Assis e, de maneira especial, à minha avó Maria Emília Santos, que faleceu neste ano. A valorização do conhecimento na minha família, e portanto a possibilidade da vida que possuo, provêm diretamente da visão e esforço destas mulheres.
