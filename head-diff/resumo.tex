%%%%%%%%%%%%%%%%%%%%%%%%%%%%%%%%%%%%%%%%%%%%%%%%%%%%%%%%%%%%%%%%%%%
%                                                                 %
%                            RESUMO                               %
%                                                                 %
%%%%%%%%%%%%%%%%%%%%%%%%%%%%%%%%%%%%%%%%%%%%%%%%%%%%%%%%%%%%%%%%%%%

\capitulo{Resumo}

Neste trabalho é implementada uma rede neural recorrente, como estipulada por Elman, em cujo interior são utilizadas operações de diferenças finitas para a modelagem de dados sísmicos pela solução da equação \DIFaddbegin \DIFadd{da }\DIFaddend onda acústica. A implementação foi realizada num ambiente de diferenciação automática. O treinamento de tal rede é o processo de inversão dos parâmetros físicos do meio propagado. Na Geofísica, este procedimento é chamado de inversão de forma completa da onda, sendo tradicionalmente implementado por meio do método adjunto. A diferenciação automática, por outro lado, tem por vantagem ser prontamente utilizável em inversões quaisquer. A reformulação desta inversão como o treinamento de uma rede neural possibilita novas interações do método com o universo do aprendizado de máquina. A inversão implementada via rede neural recorrente é testada através do modelo Marmousi e os métodos SGD, Momento e Adam, comuns ao treinamento de redes neurais, são utilizados possibilitando a obtenção de resultados bastante satisfatórios, quando comparadas ao modelo verdadeiro. Como ainda é difícil encontrar materiais em português que introduzam as redes neurais e as \DIFdelbegin \DIFdel{amarram }\DIFdelend \DIFaddbegin \DIFadd{amarrem }\DIFaddend aos processos de modelagem acústica, um objetivo primordial deste trabalho é servir como material didático\DIFaddbegin \DIFadd{, }\DIFaddend que resume estes tópicos para estudos futuros.
